\section{Evaluation}
\subsection{Participants}
We recruied 17 participant group (6 group of people (M=2.33), 11 individuals (7 male, 4 female)).

\subsection{Evaluation Properties}
\subsection{Quantitative Analyses}

1. What reaction did pedestrians show?
Pedestrians whose attention has been drawn by the display in general first stopped and after a short pause walked straight up to the display to inspect it in detail.

2. How did the pedestrians recognize the display?
If no one was near the display:
From far to near: Illumination -> Blinking -> Color -> Icon
If people were already standing at the display:
Honeypot Effect (Quote Prof. Alt!)

3. Interpretation of the display
20\% of the users got the interpretation right on the first shot. Most users in total guessed in the right direction of the original intention - that either a crosswalk is nearby or the display shall fulfill some kind of security feature to warn about approaching vehicles. 
Some users confused the display as an art installation.

4. Do you think, the display makes sense? Any surplus value recognized? 
Most people agreed on the usefulness of the kind of display to improve the overall security for pedestrians in traffic situations. Since the technique is new people have to get familiar with it before they rely on it. The technology has to establish and thereby prove to be save and useful.

5. Is it a security risk for pedestrians?
People see a security risk for pedestrians in the display if the technology is too young and not 100% reliable and tested. Further some people were concerned that, as long as the technology is new, people could be confused and distracted.

6. Does it improve security for pedestrians?
The illuminated color coding makes people improves awareness of the street situation, especially during night time. 

7. Did it extend your attention to traffic?
The blinking and novel nature of the display made people aware of the street situation.

8. Would you use such a display?
People would use the display if it were an established technology but primarily rely on their own senses. Concerns expressed regarded technical and security issues.

9. Do you think the display is unnecessary?
Users were torn regarding the necessity of the display. Some thought the existing traffic lights are sufficient, some agreed on the purpose of improving security next to view-blocking cars.

10. Would you accept it, if it serves road safety? 
All users agreed under the condition the system actually works.

11. Would you provide it in your personal car, if it serves road safety?
Most users would provide the display in their own car if it is free or offered as a standard feature in cars and to do something good for society.

12. Would you provide it in your personal car for other purposes?
Half the people would NOT like arbitrary content to be displayed on their personal car.
Reasons where disliking of the idea, privacy issues and fear of advertisement.
In contrary the other half would like the idea just because of advertisement and the idea of making some money with advertisement on their car.

13. Which other contents could be displayed on such a display?
Traffic news, Nearby public transportation connections, objects of interest like cafes or bars, News feeds, advertisement, personal messages to other pedestrians, movies.


\subsection{Qualitative Analyses}

statistical package/ program
Description of the statistical procedures
bla EXAMPLE: ... We have used means and standard deviations to represent the average and typical spread of values of variables. We have shown the precision of our estimates of outcome statistics as 95% confidence limits (which define the likely range of the true value in the population from which we drew our sample). The p values shown represent the probability of a more extreme absolute value than the observed value of the effect if the true value of the effect was zero or null. Statistically significant effects are those for which the zero or null value of the effect lies outside the 95% confidence interval (i.e., p < 0.05).
